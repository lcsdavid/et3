pgm est une application permettant de manipuler des fichiers .pgm avec plusieurs opérations.

On peut recevoir une aide en exécutant l\textquotesingle{}application avec l\textquotesingle{}option suivante \+: {\ttfamily \$ ./pgm -\/-\/help}

\tabulinesep=1mm
\begin{longtabu} spread 0pt [c]{*{3}{|X[-1]}|}
\hline
\rowcolor{\tableheadbgcolor}\textbf{ Opérations  }&\textbf{ Commande  }&\textbf{ Variables   }\\\cline{1-3}
\endfirsthead
\hline
\endfoot
\hline
\rowcolor{\tableheadbgcolor}\textbf{ Opérations  }&\textbf{ Commande  }&\textbf{ Variables   }\\\cline{1-3}
\endhead
Adapter le contraste  &{\ttfamily -\/ac, -\/-\/adapt-\/contrast}  &{\ttfamily \mbox{[}V\+A\+L\+UE\mbox{]} \mbox{[}F\+I\+LE\mbox{]} (entrée) (\mbox{[}F\+I\+LE\mbox{]} (sortie))}   \\\cline{1-3}
Noir et blanc  &{\ttfamily -\/bw, -\/-\/black-\/n-\/white}  &{\ttfamily \mbox{[}F\+I\+LE\mbox{]} (entrée) (\mbox{[}F\+I\+LE\mbox{]} (sortie))}   \\\cline{1-3}
Closing  &{\ttfamily -\/cl, -\/-\/closing}  &{\ttfamily \mbox{[}F\+I\+LE\mbox{]} (entrée) (\mbox{[}F\+I\+LE\mbox{]} (sortie))}   \\\cline{1-3}
Copie  &{\ttfamily -\/cp, -\/-\/copy}  &{\ttfamily \mbox{[}F\+I\+LE\mbox{]} (entrée) (\mbox{[}F\+I\+LE\mbox{]} (sortie))}   \\\cline{1-3}
Dilatation  &{\ttfamily -\/d, -\/-\/dilation}  &{\ttfamily \mbox{[}F\+I\+LE\mbox{]} (entrée) (\mbox{[}F\+I\+LE\mbox{]} (sortie))}   \\\cline{1-3}
Erosion  &{\ttfamily -\/e, -\/-\/erosion}  &{\ttfamily \mbox{[}F\+I\+LE\mbox{]} (entrée) (\mbox{[}F\+I\+LE\mbox{]} (sortie))}   \\\cline{1-3}
Histogramme  &{\ttfamily -\/h, -\/-\/histogram}  &{\ttfamily \mbox{[}F\+I\+LE\mbox{]} (entrée) (\mbox{[}F\+I\+LE\mbox{]} (sortie))}   \\\cline{1-3}
Opening  &{\ttfamily -\/o, -\/-\/opening}  &{\ttfamily \mbox{[}F\+I\+LE\mbox{]} (entrée) (\mbox{[}F\+I\+LE\mbox{]} (sortie))}   \\\cline{1-3}
\end{longtabu}


Par défaut les images seront stockés dans un dossier (out/) à la racine de l’exécutable. Si une sortie est précisé l\textquotesingle{}image sera stocké à la destination indiqué sous réserve de la validité de la sortie fourni. Dans le cas ou la sortie spécifié nécessite de créer de nouveaux dossiers, ils seront créés automatiquement.

\subsection*{Compilation }

Pour compiler la seul chose à faire est d\textquotesingle{}exécuter la commande suivante\+: {\ttfamily make}

Le Makefile compile tout le contenu du dossier src. Il prend en compte quel est le système d\textquotesingle{}exploitation pour changer la cible\+: {\ttfamily pgm $\vert$ pgm.\+exe $\vert$ pgm.\+app}

L\textquotesingle{}application est compilable sur toute plateforme\+: Windows, Unix (Linus, OS X (à vérifier)).

\subsection*{Documentation }

Il y a une \href{doxy/html/index.html}{\tt documentation} généré sous Doxygen dans le dossier \href{doxy/}{\tt doxy}.

\subsection*{Erreur dans les fichiers .pgm }

Si le fichier comporte une erreur dans la majorité des cas l\textquotesingle{}application indique que le fichier est corrompu. Dans certain cas, l\textquotesingle{}application va indiqué à l\textquotesingle{}utilisateur qu\textquotesingle{}il va essayer tant bien que mal de \char`\"{}deviner\char`\"{} où est l\textquotesingle{}erreur dans l\textquotesingle{}image et ainsi la corrigé.

\subsection*{Fichier }

Les sources et headers sont stockés dans le dossier src. Les objects dans le dossier obj.

\tabulinesep=1mm
\begin{longtabu} spread 0pt [c]{*{2}{|X[-1]}|}
\hline
\rowcolor{\tableheadbgcolor}\textbf{ Sources  }&\textbf{ Headers   }\\\cline{1-2}
\endfirsthead
\hline
\endfoot
\hline
\rowcolor{\tableheadbgcolor}\textbf{ Sources  }&\textbf{ Headers   }\\\cline{1-2}
\endhead
\mbox{\hyperlink{main_8c}{main.\+c}}  &\\\cline{1-2}
\mbox{\hyperlink{image_8c}{image.\+c}}  &\mbox{\hyperlink{image_8h}{image.\+h}}   \\\cline{1-2}
\mbox{\hyperlink{histogram_8c}{histogram.\+c}}  &\mbox{\hyperlink{histogram_8h}{histogram.\+h}}   \\\cline{1-2}
\mbox{\hyperlink{utility_8c}{utility.\+c}}  &\mbox{\hyperlink{utility_8h}{utility.\+h}}   \\\cline{1-2}
\end{longtabu}


\subsection*{Auteurs }

\href{mailto:lucas.david@u-psud.fr}{\tt Lucas David} \+: E\+T3 Info

\href{mailto:robin.matha@u-psud.fr}{\tt Robin Matha}\+: E\+T3 P\+SO 